% Resume autogenerated using resume-latex.xsl

\documentclass[a4]{resume}
\usepackage{paralist}
\definecolor{ruleendcolor}{rgb}{0.6, 0.6, 0.6}


    
\author{Haojun Bao}

    
\email{baohaojun@gmail.com}

    
\phone{13693578957}

    
\webpage{http://windows-config.googlecode.com}

  

\begin{document}
\maketitle

\section{Education}

\affiliation[Bachelor of Arts in Control Theory and Engineering]
            {Zhejiang University}
            {September 1997-July 2001}

\affiliation[Master of Science in Control Theory and Engineering]
            {Institute of Automation, Chinese Academy of Sciences}
            {September 2001-July 2004}

\section{Experience}

\affiliation[Senior Software Engineer]
            {Borqs (\url{http://www.borqs.com})}
            {November 2008-present}

\begin{compactitem}

	\item Worked with the Tools team of Borqs, a mobile device
          software company that developped the OMS, based on Google
          Android. \par

	\item Created Borqs Engineering Tool in C++ (VC6, MFC). It
          provides scriptable access to testing commands of the
          phone. Main features include: batch testing, auto
          connection, connecting through either usblan or ADB, phone
          screen framebuffer capture. The GUI of this tool has a
          keybinding that mimics Emacs/GNU readline, with a command
          history like Bash. \par

        \item Created fb2bmp in C (arm-linux-gcc). It captures the
          phone frame buffer, and save it as a .bmp file. It's
          integrated into the Borqs Engineering Tool. \par

        \item Created Engineering Tool Library. So that our customers
          can develop their own tools based on this library. \par

        \item Created Service Tool. It's a tool for the repair center
          of Borqs partners. Based on Engineering Tool Library. \par

        \item Created SerialPort Usblan Passthrough Tool in Python. It
          provides access to phone internal UART through usblan. Open
          source project ``com0com'' and ``PySerial'' is used. Has a
          GUI implemented in PyQt. \par

        \item Worked on Flashing Tool (VC++) and Release Tool (Bash
          script, Python). Enable developper to flash software onto
          the phone with ``one click'' away. Completely rewrote Flashing
          Tool to fix its multithread bugs and provide more user
          friendly usage.

        \item Mentored junior software engineers of their software
          development, open source tools usage. Mentored an intern to
          use Python and PyQt to develop platform independent version
          of Borqs Engineering Tool.
      
\end{compactitem}

\affiliation[Software Engineer]
            {Mobile Device/Global Software Group, Motorola \par}
            {September 2005-September 2008}

\begin{compactitem}

	\item Worked on phone multimedia software autotesting tool development.

        \item Worked on phone multimedia software development.

\end{compactitem}

\affiliation[Software Engineer]
            {Qilin Software \par}
            {October 2004-September 2005}
\begin{compactitem}
  \item Worked on CRM software testing
\end{compactitem}

\affiliation[Open Source Project]
            {Windows+Cygwin+Linux \par (\url{http://windows-config.googlecode.com})}
            {May 2005-Now}
            
\begin{compactitem}
  \item Created ``scim-fcitx'' (C++, Linux), a GNU/Linux input method
    based on SCIM and Fcitx. \par

  \item Created ``sdim'' (C++, Python, Win32), a Win32 input
    method. First wrote it in C++ completely; then refactored it into
    C++ + Python with a client/server model, to improve modularity and
    debugging easiness. \par

  \item Created ``personal windows manager'' (C++, Python), a win32
    windows manager extension, inspired by Sawfish, BlackBox, BB4Win. \par

  \item Created ``regtools'' (Python, pywin32), tools to manipulate
    win32 registry with regular expression search and replace. \par

  \item Created ``RunBhjRun'' (C++), a win32 ``run dialog''
    replacement. Mimics Emacs key-binding, better completion, better
    history. \par

  \item Created ``grepall.py'', a Python script using ``find(1)'' and
    ``grep(1)'', for easier and quicker source code browsing. \par

  \item Created ``ImapMonitor'' (Python, PyQt), a script to monitor my
    Email. Inspired by xmailbox.\par

  \item Other small scripts/programs, all in \url{http://windows-config.googlecode.com}. \par

  
    

\end{compactitem}


\section{Skills}


\begin{compactitem}

      \item Programming in languages including C++, C, Python, Bash, a little Perl and a little Emacs Lisp. \par

      \item Emacs. \par

      \item GNU/Linux, Cygwin. \par
    
\end{compactitem}


\end{document}
