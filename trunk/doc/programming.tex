% This is the file GB.tex of the CJK package
%   for testing Chinese (in GB encoding).
%
% written by Werner Lemberg <wl@gnu.org>
%
% Version 4.8.0 (22-May-2008)

\documentclass[10pt]{article}

\usepackage{CJK}
\usepackage{parskip}

\begin{document}
\begin{CJK*}{UTF8}{simsun}
\CJKtilde

\title{程序员的矛盾论}
\date{}
\author{包昊军}
\maketitle

\section{}

矛盾无处不在。矛盾就是江湖。只要有人的地方,就有江湖。只要有人的地方,
也就有矛盾。做为一个程序员,很不幸,我们也是人。而不是神。只有Linus那样
的人才敢公然宣称自己是神。别人没了他不行。结果还是让很多人不舒服了。

(如果不相信Linus自己说过他是神,请Google “Linus God”)。

程序员是干什么的?写程序的。程序本来没有,程序员把它写出来,程序就有了。
有和没有,是最经典的矛盾。

程序员写出来的程序,有的很好,有的很烂。矛盾。

程序员写程序要用编程语言,编程语言有很多种,有C,C++,选C还是C++呢?矛
盾。还有Java,Python,Perl,Bash,Sed,Awk,Lisp,Ruby,汇编。纸带打孔。

0 和 1。

程序员都需要花时间学习,写程序也要花时间。学得太多了,会导致写程序的时
间少了,相应的写的程序也少了,于是水平可能就不会有写过很多程序的人高。

写程序的时间多了,花在学习上的时间就少了,新知识少了,写的程序可能会没
有新灵感,于是变成一件没有创造力的重复劳动。



只懂一门编程语言的人,不可能是一个优秀的程序员。


\end{CJK*}

\end{document}

%%% Local Variables:
%%% coding: utf-8
%%% mode: latex
%%% TeX-master: t
%%% End:
